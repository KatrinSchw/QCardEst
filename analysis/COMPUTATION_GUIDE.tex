\documentclass{article}
\usepackage[utf8]{inputenc}
\usepackage{amsmath}
\usepackage{listings}
\usepackage{xcolor}
\usepackage{booktabs}
\usepackage{hyperref}

\title{QCardEst/QCardCorr: Key Computation Functions Guide}
\author{QCardEst/QCardCorr}
\date{\today}

\begin{document}

\maketitle

The difference between QCardEst and QCardCorr is controlled by the \texttt{valueType} setting:

\begin{itemize}
\item \textbf{QCardEst}: \texttt{valueType = "rows"} $\rightarrow$ Direct cardinality prediction
\item \textbf{QCardCorr}: \texttt{valueType = "rowFactor"} $\rightarrow$ Correction factor prediction (multiplies existing prediction)
\end{itemize}

\section{Entry Point: Main Execution Flow}

\textbf{File}: \texttt{runRegression.py}

\textbf{Key Function}: Main execution

\begin{lstlisting}[language=Python, basicstyle=\small\ttfamily]
# Lines 49-65: Setup and execution
env = CardEnv(inputFile=settings["data"] + ".csv", settings=settings)
agent = interpretation.fromString(settings["loss"], None, env)
model = vqc(settings=settings, nInputs=env.getInputSize(), nOutputs=agent.nInputs, norm=False)
optimizer = Regression(agent, env, settings)
optimizer.run()  # Main training loop
optimizer.listSolutions()  # Generate final predictions
\end{lstlisting}

\textbf{Key Input}: \texttt{settings["valueType"]} determines Est vs Corr mode

\section{Training Loop: Core Learning Process}

\textbf{File}: \texttt{ML/GradientQML.py}

\textbf{Key Function}: \texttt{GradientQML.run()} (lines 43-80)

\textbf{Critical Computation Steps}:

\begin{lstlisting}[language=Python, basicstyle=\small\ttfamily]
# Line 55: Get training sample
state, expected = self.env.step()

# Line 56: Forward pass through quantum model
prediction = self.agent(Tensor(state))

# Line 57: Compute loss
closs, logState = self.interpret(state, expected, prediction, logState)

# Lines 63-65: Backward pass and optimization
self.optimizer.zero_grad()
loss.backward()
self.optimizer.step()

# Lines 72-74: Periodic evaluation and logging
stats = self.env.elvaluateModel(self.agent)
self.resultFile.write(",".join(str(e) for e in ([episode] + stats)))
\end{lstlisting}

\textbf{Key Input}: 
\begin{itemize}
\item \texttt{state}: Feature vector (query tables + selectivities)
\item \texttt{expected}: Target value (computed based on \texttt{valueType})
\end{itemize}

\section{Expected Value Computation: Est vs Corr Difference}

\textbf{File}: \texttt{cardEnv.py}

\textbf{Key Function}: \texttt{CardEnv.expectedWithType()} (lines 34-39)

\textbf{Critical Logic}:

\begin{lstlisting}[language=Python, basicstyle=\small\ttfamily]
def expectedWithType(self, entry, type):
    if type in ["rowFactor", "rowsFactor"]:  # QCardCorr
        return self.expectedMapping(entry["rows"] / entry["rowsPredicted"])
    if type in ["costFactor", "costsFactor"]:
        return self.expectedMapping(entry["cost"] / entry["costPredicted"])
    return self.expectedMapping(entry[type])  # QCardEst: direct value

def expectedMapping(self, value):
    return math.log(value)  # Log-space transformation
\end{lstlisting}

\textbf{Key Difference}:
\begin{itemize}
\item \textbf{QCardEst} (\texttt{"rows"}): \texttt{expected = log(true\_cardinality)}
\item \textbf{QCardCorr} (\texttt{"rowFactor"}): \texttt{expected = log(true\_cardinality / predicted\_cardinality)}
\end{itemize}

\section{Model Evaluation: Metrics Computation}

\textbf{File}: \texttt{cardEnv.py}

\textbf{Key Function}: \texttt{CardEnv.elvaluateModel()} (lines 87-139)

\textbf{Critical Computation} (lines 108-135):

\begin{lstlisting}[language=Python, basicstyle=\small\ttfamily]
for i, prediction in enumerate(predictions):
    entry = evaluationSet[i]
    value = model.predict(prediction).item()  # Model output
    expected = self.expected(entry)  # True value
    
    # Compute error metrics
    diffs.append(abs(value - expected))
    factors.append(compute_factor(value, expected))
    
    # Convert to cardinality space
    if "Factor" in self.valueType:  # QCardCorr
        correction = math.exp(value)
        card = math.log(correction * entry["rowsPredicted"])
    else:  # QCardEst
        card = value
    
    expectedCard = math.log(entry["rows"])
    diffCards.append(abs(card - expectedCard))
    factorCards.append(compute_factor(card, expectedCard))

# Return statistics: [mean_diff, median_diff, var_diff, mean_factor, ...]
return [stat.mean(diffs), stat.median(diffs), ...]
\end{lstlisting}

\textbf{Key Output}: 13-element statistics array written to CSV (line 73 in GradientQML.py)

\section{Final Predictions: Solution Generation}

\textbf{File}: \texttt{cardEnv.py}

\textbf{Key Function}: \texttt{CardEnv.listSolutions()} (lines 141-155)

\textbf{Critical Computation}:

\begin{lstlisting}[language=Python, basicstyle=\small\ttfamily]
for index, entry in enumerate(self.data):
    state = Tensor(entry["features"])
    temp = model(state)  # Quantum model output
    prediction = model.predict(temp)  # Post-processed prediction
    expected = self.expected(entry)  # True value
    loss = model.loss(temp, expected)  # Loss value
    
    # Compute factor (error metric)
    factor = abs(prediction / expected)
    if factor < 1 and factor != 0:
        factor = 1 / factor
    
    result.append([index, prediction.item(), expected, factor.item(), loss.item()])
\end{lstlisting}

\textbf{Key Output}: CSV file with columns: \texttt{id, prediction, expected, factor, loss}

\section{Quantum Model Architecture}

\textbf{File}: \texttt{ML/models.py}

\textbf{Key Function}: \texttt{vqc()} (lines 31-87)

\textbf{Critical Components}:

\begin{lstlisting}[language=Python, basicstyle=\small\ttfamily]
# Lines 44-49: Create parameterized quantum circuit
qc = circuits.parametrizedCircuit(nQubits, settings)
X = list(qc.parameters)[:divider]  # Input parameters
params = list(qc.parameters)[divider:]  # Trainable parameters

# Lines 63-67: Connect quantum circuit to PyTorch
qnn = CircuitQNN(qc, input_params=X, weight_params=params, quantum_instance=qi)
quantumNN = TorchConnector(qnn, initialWeights)

# Lines 70-85: Build complete model
model = torch.nn.Sequential(
    paddingLayer,      # Pad input to qubit count
    quantumNN,          # Quantum neural network
    ReshapeSumLayer,   # Reduce quantum states to outputs
    NormLayer          # Normalize probabilities
)
\end{lstlisting}

\textbf{Key Input}: \texttt{settings} dict with:
\begin{itemize}
\item \texttt{encoding}: \texttt{["rx", "rz"]} (feature encoding)
\item \texttt{reps}: Number of layers
\item \texttt{calc}: \texttt{"yz"} (measurement basis)
\item \texttt{entangleType}: \texttt{"circular"} (entanglement pattern)
\end{itemize}

\section{Post-Processing: Interpretation Layers}

\textbf{File}: \texttt{ML/regressionInterpretation.py}

\textbf{Key Function}: \texttt{fromString()} (lines 244-274)

\textbf{Available Interpretations}:
\begin{itemize}
\item \texttt{"linear"}: \texttt{LinearScale} - Single probability with scaling
\item \texttt{"rational"}: \texttt{Rational} - Ratio of two probabilities
\item \texttt{"rationalLog"}: \texttt{RationalLog} - Log of ratio
\item \texttt{"threshold"}: \texttt{SecondValueThreshold} - Threshold-controlled
\item \texttt{"PlaceValue"}: \texttt{PlaceValueSystem} - Place-value encoding
\end{itemize}

\textbf{Key Function}: \texttt{Interpretation.predict()} (line 32)

\begin{lstlisting}[language=Python, basicstyle=\small\ttfamily]
def predict(self, prediction: torch.Tensor):
    """Convert quantum output to scalar prediction"""
    return prediction
\end{lstlisting}

\textbf{Key Function}: \texttt{Interpretation.loss()} (line 38)

\begin{lstlisting}[language=Python, basicstyle=\small\ttfamily]
def loss(self, prediction: torch.Tensor, compare):
    """MSE loss for optimization"""
    return (self.predict(prediction) - compare)**2
\end{lstlisting}

\section{Data Loading and Feature Extraction}

\textbf{File}: \texttt{cardEnv.py}

\textbf{Key Function}: \texttt{CardEnv.load\_data()} (lines 47-78)

\textbf{Data Format} (from \texttt{costs/*.csv}):
\begin{enumerate}
\item Tablenames (separated by ';')
\item Selectivities (separated by ';')
\item PostgreSQL cost
\item PostgreSQL/MSCN cardinality prediction
\item Actual execution time
\item True cardinality
\end{enumerate}

\textbf{Key Processing}:

\begin{lstlisting}[language=Python, basicstyle=\small\ttfamily]
# Lines 56-67: Extract features
query = tmp[0].split(";")
selectivities = tmp[1].split(";")
features = [[table_index, selectivity] for table, sel in zip(query, selectivities)]

# Line 76: Create data entry
datapoint = {
    "id": id,
    "features_raw": features,
    "features": self.feature_map(features),  # Map to [0, π]
    "rows": int(values[3]),  # True cardinality
    "rowsPredicted": int(values[1]),  # Classical prediction
    "cost": float(values[2]),
    "costPredicted": float(values[0])
}
\end{lstlisting}

\section{Result File Format}

\textbf{Training Logs}: \texttt{results/*.csv}

\textbf{Format}: \texttt{episode, mean\_diff, median\_diff, var\_diff, mean\_factor, median\_factor, var\_factor, mean\_diffCards, median\_diffCards, var\_diffCards, mean\_factorCards, median\_factorCards, var\_factorCards, close}

\textbf{Generated by}: \texttt{GradientQML.run()} line 73

\textbf{Solutions}: \texttt{results/solutions/*.sl.csv}

\textbf{Format}: \texttt{id, prediction, expected, factor, loss}

\textbf{Generated by}: \texttt{CardEnv.listSolutions()} line 154

\section{Summary: Key Files for Results Computation}

\begin{enumerate}
\item \textbf{\texttt{runRegression.py}}: Entry point, settings configuration
\item \textbf{\texttt{ML/GradientQML.py}}: Training loop, optimization, result logging
\item \textbf{\texttt{cardEnv.py}}: 
\begin{itemize}
\item \texttt{expectedWithType()}: Est vs Corr difference
\item \texttt{elvaluateModel()}: Metrics computation
\item \texttt{listSolutions()}: Final predictions
\end{itemize}
\item \textbf{\texttt{ML/models.py}}: Quantum model architecture (\texttt{vqc()})
\item \textbf{\texttt{ML/regressionInterpretation.py}}: Post-processing layers
\item \textbf{\texttt{cardEnv.py} (load\_data)}: Data loading and feature extraction
\end{enumerate}

\section{Key Differences: QCardEst vs QCardCorr}

\begin{table}[h]
\centering
\begin{tabular}{lcc}
\toprule
Aspect & QCardEst (\texttt{valueType="rows"}) & QCardCorr (\texttt{valueType="rowFactor"}) \\
\midrule
\textbf{Expected Value} & \texttt{log(true\_cardinality)} & \texttt{log(true\_cardinality / predicted\_cardinality)} \\
\textbf{Model Output} & Direct cardinality in log space & Correction factor in log space \\
\textbf{Final Cardinality} & \texttt{exp(prediction)} & \texttt{exp(prediction) * predicted\_cardinality} \\
\textbf{Use Case} & Predict from scratch & Correct existing predictions \\
\bottomrule
\end{tabular}
\end{table}

\end{document}

