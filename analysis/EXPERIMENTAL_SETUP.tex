\documentclass{article}
\usepackage[utf8]{inputenc}
\usepackage{amsmath}
\usepackage{listings}
\usepackage{xcolor}
\usepackage{booktabs}
\usepackage{hyperref}

\title{Experimental Setup: Where Parameters Are Defined}
\author{QCardEst/QCardCorr}
\date{\today}

\begin{document}

\maketitle

This document shows exactly where each experimental setup parameter is defined in the codebase.

\section{Overview}

The experimental setup parameters are configured in the \textbf{\texttt{settings} dictionary} in \texttt{runRegression.py}, which can be:
\begin{enumerate}
\item \textbf{Set directly} in the file (lines 14-36)
\item \textbf{Overridden via command line} (lines 38-40)
\item \textbf{Automatically determined} from data (for some parameters)
\end{enumerate}

\section{Query Characteristics: Up to 6 Joined Tables}

\subsection{Where Defined:}
\begin{itemize}
\item \textbf{Primary}: \texttt{cardEnv.py}, line 50
\item \textbf{Source}: Automatically extracted from data file
\end{itemize}

\begin{lstlisting}[language=Python, basicstyle=\small\ttfamily]
# cardEnv.py, line 48-50
queryInfo = extractTablesFromCardsCSVfile(file)
self.setMaxId(queryInfo["nTables"] - 1)
self.numFeatures = queryInfo["maxQuerySize"]  # ← Up to 6 tables
\end{lstlisting}

\textbf{Data Source}: 
\begin{itemize}
\item \texttt{costs/stats/statsCards6.csv} $\rightarrow$ max 6 tables
\item \texttt{costs/stats/stats4.csv} $\rightarrow$ exactly 4 tables
\item \texttt{costs/jobSimple/job.csv} $\rightarrow$ variable
\end{itemize}

\textbf{Manual Override}:
\begin{itemize}
\item \texttt{runRegression.py}, line 35: \texttt{"numFeatures": 6} (can be set manually)
\end{itemize}

\section{Quantum Model: 6 Qubits}

\subsection{Where Defined:}
\begin{itemize}
\item \textbf{Primary}: \texttt{ML/models.py}, lines 39-41
\item \textbf{Calculation}: \texttt{nQubits = max(nInputs, nQubitsOut)}
\end{itemize}

\begin{lstlisting}[language=Python, basicstyle=\small\ttfamily]
# ML/models.py, lines 39-41
nQubitsOut = math.ceil(math.log2(nOutputs))  # Output qubits needed
nQubits = max(nInputs, nQubitsOut)  # ← 6 qubits (from numFeatures=6)
\end{lstlisting}

\textbf{Determined by}:
\begin{itemize}
\item \texttt{nInputs} = \texttt{env.getInputSize()} = \texttt{numFeatures} = 6
\item \texttt{nOutputs} = \texttt{agent.nInputs} (depends on interpretation layer, typically 2)
\end{itemize}

\textbf{Result}: With \texttt{numFeatures=6} and typical \texttt{nOutputs=2}, we get \texttt{nQubits = max(6, 1) = 6}

\section{VQC Layers: 16 Layers}

\subsection{Where Defined:}
\begin{itemize}
\item \textbf{Primary}: \texttt{runRegression.py}, line 18 (default: 6, but overridden to 16)
\item \textbf{Used in}: \texttt{ML/circuits.py}, line 86
\end{itemize}

\begin{lstlisting}[language=Python, basicstyle=\small\ttfamily]
# runRegression.py, line 18
settings = {
    "reps": 6,  # Default, but typically overridden to 16
    ...
}

# ML/circuits.py, line 86
processing = variationalLayers(nQubits, vqcConf, maxLayers=vqcConf.reps)  # ← reps=16
\end{lstlisting}

\textbf{How to Set}:

\begin{lstlisting}[language=bash, basicstyle=\small\ttfamily]
# Via command line
python runRegression.py '{"reps": 16}'

# Or in runRegression.py directly
settings = {"reps": 16, ...}
\end{lstlisting}

\textbf{Where Used}:
\begin{itemize}
\item \texttt{ML/circuits.py:parametrizedCircuit()} $\rightarrow$ Creates 16 variational layers
\item Each layer contains: encoding gates (rx, rz) + entangling gates (cx)
\end{itemize}

\section{Gates Configuration}

\subsection{Where Defined:}
\begin{itemize}
\item \textbf{Primary}: \texttt{runRegression.py}, lines 16, 20-21
\item \textbf{Processed in}: \texttt{ML/circuits.py}, lines 24-36
\end{itemize}

\subsection{Encoding Gates: \texttt{["rx", "rz"]}}

\begin{lstlisting}[language=Python, basicstyle=\small\ttfamily]
# runRegression.py, line 16
settings = {
    "encoding": ["rx", "rz"],  # ← Encoding gates
    ...
}
\end{lstlisting}

\textbf{Used in}: \texttt{ML/circuits.py:encodingCircuit()} (line 39)
\begin{itemize}
\item Creates rotation gates: \texttt{qc.rx()} and \texttt{qc.rz()} for each qubit
\end{itemize}

\subsection{Entangling Gates: \texttt{"cx"} (CNOT)}

\begin{lstlisting}[language=Python, basicstyle=\small\ttfamily]
# runRegression.py, line 21
settings = {
    "entangle": "cx",  # ← Entangling gate type
    ...
}
\end{lstlisting}

\textbf{Used in}: \texttt{ML/circuits.py:variationalLayers()} (lines 115-118)

\begin{lstlisting}[language=Python, basicstyle=\small\ttfamily]
# Circular entanglement pattern
qc.cx(nQubits - 1, 0)  # Last to first
for qubit in range(nQubits - 1):
    qc.cx(qubit, qubit + 1)  # Circular chain
\end{lstlisting}

\subsection{Entanglement Type: \texttt{"circular"}}

\begin{lstlisting}[language=Python, basicstyle=\small\ttfamily]
# runRegression.py, line 20
settings = {
    "entangleType": "circular",  # ← Entanglement pattern
    ...
}
\end{lstlisting}

\subsection{Measurement Basis: \texttt{"yz"}}

\begin{lstlisting}[language=Python, basicstyle=\small\ttfamily]
# runRegression.py, line 19
settings = {
    "calc": "yz",  # ← Measurement basis (Y and Z Pauli measurements)
    ...
}
\end{lstlisting}

\textbf{Processed in}: \texttt{ML/circuits.py:VQCsettings.\_\_init\_\_()} (line 31)

\begin{lstlisting}[language=Python, basicstyle=\small\ttfamily]
self.calcGates = stringToCircuitList(settings["calc"])  # ["y", "z"]
\end{lstlisting}

\section{Training: 8000 Optimization Episodes}

\subsection{Where Defined:}
\begin{itemize}
\item \textbf{Primary}: \texttt{runRegression.py}, line 23 (default: 1, but overridden to 8000)
\item \textbf{Used in}: \texttt{ML/GradientQML.py}, line 50
\end{itemize}

\begin{lstlisting}[language=Python, basicstyle=\small\ttfamily]
# runRegression.py, line 23
settings = {
    "numEpisodes": 1,  # Default, but typically overridden to 8000
    ...
}

# ML/GradientQML.py, line 50
for episode in range(self.numEpisodes):  # ← 8000 episodes
    ...
\end{lstlisting}

\textbf{How to Set}:

\begin{lstlisting}[language=bash, basicstyle=\small\ttfamily]
# Via command line
python runRegression.py '{"numEpisodes": 8000}'

# Or in runRegression.py directly
settings = {"numEpisodes": 8000, ...}
\end{lstlisting}

\textbf{Where Used}:
\begin{itemize}
\item \texttt{ML/GradientQML.py:run()} $\rightarrow$ Main training loop (lines 50-75)
\item Each episode: forward pass, loss computation, backward pass, optimization step
\end{itemize}

\section{Optimizer: Adam with Learning Rate Decay}

\subsection{Where Defined:}
\begin{itemize}
\item \textbf{Primary}: \texttt{runRegression.py}, lines 24-25
\item \textbf{Configured in}: \texttt{ML/GradientQML.py}, lines 22-38
\end{itemize}

\subsection{Optimizer Type: \texttt{"Adam"}}

\begin{lstlisting}[language=Python, basicstyle=\small\ttfamily]
# runRegression.py, line 24
settings = {
    "optimizer": "Adam",  # ← Optimizer type
    ...
}
\end{lstlisting}

\textbf{Configured in}: \texttt{ML/GradientQML.py}, lines 28-29

\begin{lstlisting}[language=Python, basicstyle=\small\ttfamily]
if self.settings["optimizer"].lower() == "adam":
    self.optimizer = Adam(self.agent.model.parameters(), lr=lr, amsgrad=True)
\end{lstlisting}

\subsection{Learning Rate Decay: \texttt{[0.01, 100, 0.9]}}

\begin{lstlisting}[language=Python, basicstyle=\small\ttfamily]
# runRegression.py, line 25
settings = {
    "lr": [0.01, 100, 0.9],  # ← [initial_lr, step_size, gamma]
    ...
}
\end{lstlisting}

\textbf{Interpretation}:
\begin{itemize}
\item \texttt{0.01}: Initial learning rate
\item \texttt{100}: Decay every 100 episodes
\item \texttt{0.9}: Multiply LR by 0.9 at each decay step
\end{itemize}

\textbf{Configured in}: \texttt{ML/GradientQML.py}, lines 35-36

\begin{lstlisting}[language=Python, basicstyle=\small\ttfamily]
if isinstance(lrSettings, list):
    self.scheduler = StepLR(self.optimizer, step_size=lrSettings[1], gamma=lrSettings[2])
    # StepLR(optimizer, step_size=100, gamma=0.9)
\end{lstlisting}

\textbf{Applied in}: \texttt{ML/GradientQML.py}, line 66

\begin{lstlisting}[language=Python, basicstyle=\small\ttfamily]
self.scheduler.step()  # Decay learning rate every episode
\end{lstlisting}

\section{Complete Settings Dictionary Example}

For the previous experimental setup, the settings would be:

\begin{lstlisting}[language=Python, basicstyle=\small\ttfamily]
# runRegression.py
settings = {
    "features": "simple",
    "encoding": ["rx", "rz"],        # Encoding gates
    "reuploading": False,
    "reps": 16,                      # ← 16 VQC layers
    "calc": "yz",                    # Measurement basis
    "entangleType": "circular",      # Entanglement pattern
    "entangle": "cx",                # Entangling gate
    "reward": "rational",
    "numEpisodes": 8000,             # ← 8000 episodes
    "optimizer": "Adam",             # ← Adam optimizer
    "lr": [0.01, 100, 0.9],         # ← LR decay: start=0.01, step=100, gamma=0.9
    "prefix": "Paper",
    "noise": 0,
    "seed": 42,
    "batchsize": 1,
    "loss": "linear",                # or "rational", "threshold", etc.
    "data": "stats/statsCards6",     # ← Up to 6 tables
    "valueType": "rows",             # or "rowFactor" for QCardCorr
    "numFeatures": 6                 # ← 6 qubits (can be auto-detected)
}
\end{lstlisting}

\section{Summary: File Locations}

\begin{table}[h]
\centering
\begin{tabular}{lcc}
\toprule
Parameter & Location & Line(s) \\
\midrule
\textbf{Up to 6 tables} & \texttt{cardEnv.py} & 50 (auto from data) \\
\textbf{6 qubits} & \texttt{ML/models.py} & 39-41 (calculated) \\
\textbf{16 VQC layers} & \texttt{runRegression.py} & 18 (\texttt{reps}) \\
\textbf{Gates (rx, rz, cx)} & \texttt{runRegression.py} & 16, 20-21 \\
\textbf{8000 episodes} & \texttt{runRegression.py} & 23 (\texttt{numEpisodes}) \\
\textbf{Adam optimizer} & \texttt{runRegression.py} & 24 (\texttt{optimizer}) \\
\textbf{LR decay} & \texttt{runRegression.py} & 25 (\texttt{lr}) \\
\textbf{LR scheduler} & \texttt{ML/GradientQML.py} & 35-36 (configured) \\
\bottomrule
\end{tabular}
\end{table}

\section{How to Run with These Settings}

\begin{lstlisting}[language=bash, basicstyle=\small\ttfamily]
# Command line (recommended for experiments)
python runRegression.py '{
    "reps": 16,
    "numEpisodes": 8000,
    "optimizer": "Adam",
    "lr": [0.01, 100, 0.9],
    "data": "stats/statsCards6",
    "numFeatures": 6,
    "prefix": "Paper"
}'
\end{lstlisting}

Or modify \texttt{runRegression.py} directly (lines 14-36).

\section{Where Settings Are Saved}

After running, the complete settings are saved to:
\begin{itemize}
\item \textbf{\texttt{results/settings/*.conf}} - Human-readable settings file
\item \textbf{Filename contains all settings} - See \texttt{ML/QML.py}, lines 34-36
\end{itemize}

Example filename:
\begin{verbatim}
Paper_simple_['rx',-'rz']_False_16_yz_circular_cx_rational_8000_Adam_[0.01,-100,-0.9]_Test_42_0_1_linear_stats-statsCards6_rows_6
\end{verbatim}

This filename encodes all settings, making it easy to identify the experimental configuration.

\end{document}

